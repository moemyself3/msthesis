\chapter{Conclusion}
\section{Variable star detection}
Eclipsing binary star detection by cross identification with VSX and GCVS was successful in all 5 data sets observed.
Other variable stars including other eclipsing binaries were detected in the same field as the original intended targets of observation.
Once the implementation of a database is complete then the extractors available on feets can analyze and rank variability. 

\section{Eclipsing binary characterization}
Only period was calculated using the Lomb-Scargle Method for finding period.
HP Aur has a much longer period than was actually observed, therefore it is expected that period analysis will result in poor
relation to the period reported by GCVS\@.
Between the two observations of PR Boo only one had a complete period that resulted in a closer match to the reported period on GCVS\@.
The large discrepancy between the periods of NY Lyr are most likely effects due to the single aperture that was selected based
on average seeing. The reported maximum magnitude of 11.16 for NY Lyr suggests that a different aperture should be used.

\subsection{Limitations on characterization}
CTMO is being upgraded to include a spectrometer for spectral data collection.
Radial velocities cannot be measured without spectral data.
For this reason eclipsing binary system mass calculations cannot be achieved at CTMO\@.
At least 2 filters are required for effective temperature calculations.
Since most data taken from the first years of CTMO is unfiltered effective temperatures cannot be calculated.

\section{Considerations for the future}
Aperture photometry relies on selecting appropriate aperture size for a given observation.
This can be systematically found by calculating the full width half maximum of a source to determine the radius of aperture in pixels.
Point spread function (PSF) photometry is a better approximation of a profile of a source taking into account the more Gaussian-like
distribution of signal across the pixels of a CCD\@.
Currently, photutils v0.6 has a PSF Photometry module that is considered experimental and not used for this reason. 

Lightcurator relies on Astrometry.net service for attaching WCS and assumes this step must be completed.
To make lightcurator more universal, this step must be made optional.
Since Astrometry.net is not a python package the user must install the Astrometry program.
The image registration software, astroalign, has the potential to attach WCS if using one reference image with proper WCS, thus
eliminating the need for software outside the python environment.
Astroalign does not currently have this feature.

Optimization can be performed by reducing the number of times a file is saved to memory to read the header.
Statistics of every data frame is performed on multiple occasions as part of intermediary steps such as source detection for 
image registration. 
Lightcurator could switch the order of the pipeline and get xy source positions and flux measurements first.
Then by using the xy pixel positions from the data frames in astroalign the source matching can be done without 
the need for running source extraction. Benchmarking of both methods will need to be performed for comparison.

Lastly, for proper long term period analysis transformations to a standard system should be performed. 
As a first order approximation for unfiltered data comparison star magnitudes using the reported Johnson V magnitude are typically used.
Given that lightcurator already can query VizieR Catalogs this upgrade should be possible. 

