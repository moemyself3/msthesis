\chapter{Introduction}\label{ch:intro}

\section{}

\section{Variable stars}
Variable stars can be classified into 2 major categories: intrinsic and extrinsic.
Intrinsic variable stars change magnitude due to physical processes inside the star, for example the pulsating variable RR Lyraes.
Extrinsic variable stars change magnitude due to external processes for example occultations and eclipses.

In order to characterize eclipsing binary systems a light curve must be made.
A light curve is a time series plot of the measured magnitude.

\section{Characterizing variability}
Using different algorithms one can distinguish the difference between the types of variable star system.

This demonstrates RR Lyrae
This demonstrates Eclipsing binaries
This demonstrates solar spots



\section{Eclipsing Binary Characterization}
Binary stars can reveal physical properties by examining the light curves.
Current models use a process called WD code.

In this thesis a modified WD code will be used from the Physics of Eclipsing Binaries (PHOEBE)

\section{O-C calculations}
Observed minus Calculated


\section{Equations of motion}



\section{WD code}


\subsection{Modified WD code}


\section{Pipeline for variable star detection}



\section{Data Acquisition}
Eclipsing binary optical data acquisition requires an observer to take time series ccd photometry of one target with an observation cadence dependent on the period.

A cadence can be determined by 

For example, the data gathered for this project was collected on multiple nights and observed on the order of hours with evenly timed frames.


\begin{equation}
dt=5
\end{equation}
