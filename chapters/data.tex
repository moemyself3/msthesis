\chapter{Data}
\section{Motivation for pipline: lightcurator}
Observing eclipsing binary systems especially of EW type require an observer to track the source on the order of hours.
This means the data produced can include various stars as well as some potentially undiscovered variable stars.
For this reason, I created a python package called lightcurator available on Github \url{https://github.com/moemyself3/lightcurator}
and on PyPI \url{https://pypi.org/project/lightcurator/} for easy installation.

\section{Pipeline for variable star detection}
\begin{enumerate}
    \item Data needs to be cleaned first.
    \item Align frames
    \item Extract sources
    \item Cross match sources with catalogs
    \item Create Database of known sources
    \item Plot individual light curves
    \item Analyze individual light curves for variability estimation
    \item Sort database given variability ranking
\end{enumerate}

\subsection{Data reduction}
\subsection{Align frames}
Frame alignment is done with a python package called astroalign written by Dr.\ Martin Beroiz~\cite{beroiz_2019} which is publicly
available on Github\footnote{For the latest version of astroalign check: \url{https://github.com/toros-astro/astroalign}}

\subsection{Extract sources}
Source extraction is done with a python package called photutils written by 

\subsection{Source cross matching}
Source cross matching is done with a python package-called 

\subsection{Analyzing variability}
Variability analysis is done with a python package called feets written by J Cabral~\cite{cabral_2018}

\section{Data Acquisition}
Eclipsing binary optical data acquisition requires an observer to take time series ccd photometry of one target with an observation cadence dependent on the period.

A cadence can be determined by 

For example, the data gathered for this project was collected on multiple nights and observed on the order of hours with evenly timed frames.


\section{Observed objects}
\begin{table}
    \centering
    \begin{tabular}[h]{l c c c}
    \toprule
    Target      & Type  & Filter    & Observation date \\ \bottomrule
    RV Vir      & M     & C         & 2017/02/23 \\ \midrule
    MQ Boo      & EB    & C         & 2017/04/26 \\ \midrule
    PR Boo      & EW    & C         & 2017/03/30 \\ \midrule
                &       & C         & 2017/04/20 \\ \midrule
                &       & C         & 2017/05/11 \\ \midrule
    Alp Boo     & ``???''   & RGB       & 2017/04/05 \\ \midrule
    EQ Uma      & EW/KW & CRGB      & 2017/04/06 \\ \midrule
    HP Aur      & EA    & C         & 2017/04/13 \\ \midrule
    NY Lyr      & EW/KW & C         & 2017/07/06 \\ \midrule
    AW Ari      & EW    & GB        & 2017/10/12 \\ \midrule
    SS Ari      & EW    & C         & 2016/11/27 \\ \midrule
                &       & RGB       & 2017/11/01 \\ \midrule
    XX LMi      & EW    & RGB       & 2018/03/20 \\ \midrule
    V467 Lyr    & EW:/KE:& CRG   & 2018/06/07 \\ \midrule
    V2793 Ori   & EA    & RGB   & 2017/11/17 \\
    \bottomrule
    \end{tabular}
    \caption{Observations of eclipsing binaries. Filters: RGB corresponds to Baader CCD RGB filters and C is unfiltered}\label{tab:observations}
\end{table}
\section{Known information of targets examined}

